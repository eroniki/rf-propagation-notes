\section{Fundamentals of Radio Wave Propagation}
When electrons move through a conductor medium, they create \gls{em}  waves that can propagate through in space.
These waves can display different characteristics depending on the fundamental properties of their wavelength.

\showFigure{img/placeholder.png}{\linewidth}{\gls{em} Spectrum}{fig:emspectrum}

\gls{tx}~\gls{rx}
Path loss is caused by dissipation of the power radiated by the transmitter as well as effects of the propagation channel.
Path loss models generally assume that path loss is the same at a given transmit-receive distance1.
Shadowing is caused by obstacles between the transmitter and receiver that absorb power.
When the obstacle absorbs all the power, the signal is blocked.

Variation due to path loss occurs over very large distances (100\em1000 meters), whereas variation due to shadowing occurs over distances proportional to the length of the obstructing object (10-100 meters in outdoor environments and less in indoor environments).
Since variations due to path loss and shadowing occur over relatively large distances, this variation is sometimes refered to as \textbf{large-scale propagation effects} or \textbf{local mean attenuation}.
