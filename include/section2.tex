\section{Fundamentals of Radio Wave Propagation}
\label{sec:fundradioprop}
This section will briefly describe the fundamentals of radio waves and their propagation mechanisms.
In order to describe radio waves, the \gls{em} waves should be described since radio waves are a small subset of \gls{em} waves.
When electrons move through a conductor medium, they create \gls{em}  waves that can propagate through in space.
These waves can display different characteristics depending on the fundamental properties of the wave,  i.e.\ their wavelength.
\Cref{fig:emspectrum} depicts the \gls{em} spectrum as a function of wavelength ($\lambda$) and frequency ($f$).
\showFigure{img/em_spectrum.jpg}{\linewidth}{Electromagnetic Spectrum}{emspectrum}

As it can be seen in \Cref{fig:emspectrum}, not only does radio waves form the lowest frequency band of the spectrum, but it also contains the most diverse frequency range amongst other \gls{em} bands.
This diversity given a rise to a need of subdividing the radio wave band into smaller bands due to the fact that the waves display significantly difference physics.
Thus, the radio waves was subcategorized into smaller bands by different standards.
However, one of the most common standard was formed by \gls{ieee}~\cite{bruder2003ieee}.
\Cref{tab:radiofreqITU} and \cref{tab:radiofreqIEEE} tabulate the categories proposed by \gls{itu} and \gls{ieee}, respectively.

\begin{table}[ht]
    \centering
    \begin{tabular}{ | c | c | c | c |  } \hline
      Band Number & Symbols & Frequency Band & Wavelength Range \\ \hline
      4 & \gls{vlf} & 3 to 30 kHz & 10 to 100 km\\
      5 & \gls{lf} & 30 to 300 kHz & 1 to 10 km\\
      6 & \gls{mf} & 300 to 3000 kHz & 100 to 1000 m\\
      7 & \gls{hf} & 3 to 30 MHz & 10 to 100 m\\
      8 & \gls{vhf} & 30 to 300 MHz & 1 to 10 m\\
      9 & \gls{uhf} & 300 to 3000 MHz & 100 to 1000 cm\\
      10 & \gls{shf} & 3 to 30 GHz & 10 to 100 cm\\
      11 & \gls{ehf} & 30 to 300 GHz & 1 to 10 cm\\
      12 & \gls{thf} & 300 to 3000 GHz & 0.1 to 1 mm\\ \hline
    \end{tabular}
    \label{tab:radiofreqITU}
    \caption{Radio Wave Channel Divisions by ITU}
\end{table}

\begin{table}[ht]
    \centering
    \begin{tabular}{ | c | c | c | } \hline
        Band Designation & Frequency Band & Wavelength Range \\ \hline
        \gls{hf} & 3 to 30 kHz & 10 to 100 km\\
        \gls{vhf} & 30 to 300 kHz & 1 to 10 km\\
        \gls{uhf} & 300 to 3000 kHz & 100 to 1000 m\\
        L Band & 3 to 30 MHz & 10 to 100 m\\
        S Band & 30 to 300 MHz & 1 to 10 m\\
        C Band & 300 to 3000 MHz & 100 to 1000 cm\\
        X Band & 3 to 30 GHz & 10 to 100 cm\\
        $K_u$ Band & 30 to 300 GHz & 1 to 10 cm\\
        K Band & 30 to 300 GHz & 1 to 10 cm\\
        $K_a$ Band & 30 to 300 GHz & 1 to 10 cm\\
        V Band & 30 to 300 GHz & 1 to 10 cm\\
        W Band & 30 to 300 GHz & 1 to 10 cm\\
        G Band (mm) & 30 to 300 GHz & 1 to 10 cm\\ \hline
    \end{tabular}
    \label{tab:radiofreqIEEE}
    \caption{Radio Wave Channel Divisions by \gls{ieee}}
\end{table}

% Amongst various applications, the radio waves were prominently formed the basis of the wireless communication technologies.

\subsection{Radio Wave Propagation Mechanisms}
\subsubsection{Reflection}
\subsubsection{Diffraction}
\subsubsection{Scattering}



\subsection{Path Loss, Shadowing and Fading}
\gls{tx}~\gls{rx}
Path loss is caused by dissipation of the power radiated by the transmitter as well as effects of the propagation channel.
Path loss models generally assume that path loss is the same at a given transmit-receive distance1.
Shadowing is caused by obstacles between the transmitter and receiver that absorb power.
When the obstacle absorbs all the power, the signal is blocked.

Variation due to path loss occurs over very large distances (100 -- 1000 meters), whereas variation due to shadowing occurs over distances proportional to the length of the obstructing object (10-100 meters in outdoor environments and less in indoor environments).
Since variations due to path loss and shadowing occur over relatively large distances, this variation is sometimes refered to as \textbf{large-scale propagation effects} or \textbf{local mean attenuation}.

% \subsection{Overview of Wireless Communication Technologies}
% Explanations about the current technologies.
% \subsubsection{Cellular Telephone Systems}
% \subsubsection{Cordless Phones}
% \subsubsection{Cellular Telephone Systems}
% \subsubsection{Wireless LANs}
% \subsubsection{Wide Area Wireless Data Services}
% \subsubsection{Fixed Wireless Access}
% \subsubsection{Paging Systems}
% \subsubsection{Satellite Networks}
% \subsubsection{Bluetooth}
% \subsubsection{HomeRF}
\subsection{Indoor Radio Wave Propagation Models}


\pagebreak
